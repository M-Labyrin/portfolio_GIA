\documentclass[12pt]{article}

\usepackage[T2A]{fontenc}
\usepackage[utf8]{inputenc}
\usepackage[russian]{babel}

\renewcommand{\familydefault}{\sfdefault}

\usepackage[left=2cm,right=2cm,
  top=2cm,bottom=2cm,bindingoffset=0cm]{geometry}

\usepackage[skip=1em]{parskip}

\author{Лабырин М. С., 1 подгруппа}
\title{Команды форматирования в \LaTeX}

\hfuzz=30pt

\begin{document}

\maketitle

\begin{center}
  % chktex-file 44
  \begin{tabular}{ || l | l || }

    \hline
    Назначение команды & Вид (написание) команды \\ \hline
    \hline

    Маркированный список &
    $\backslash begin\{itemize\}$ \\ &
    $\backslash item$ foo \\ &
    $\cdots$ \\ &
    $\backslash item$ bar \\ &
    $\backslash end\{itemize\}$ \\ \hline

    Нумерованный список &
    $\backslash begin\{enumerate\}$ \\ &
    $\backslash item$ foo \\ &
    $\cdots$ \\ &
    $\backslash item$ bar \\ & 
    $\backslash end\{enumerate\}$ \\ \hline

    Центрирование &
    $\backslash begin\{center\}$ \\ &
    lorem \ldots \\ &
    $backslash end\{center\}$ \\ &
    ИЛИ \\ &
    $\backslash centering$ \\ \hline

    Выравнивание по левому краю &
    $\backslash begin\{flushleft\}$ \\ &
    lorem \ldots \\ &
    $backslash end\{flushleft\}$ \\ &
    ИЛИ \\ &
    $\backslash raggedright$ \\ \hline

    Выравнивание по правому краю &
    $\backslash begin\{flushright\}$ \\ &
    lorem \ldots \\ &
    $backslash end\{flushright\}$ \\ &
    ИЛИ \\ &
    $\backslash raggedleft$ \\ \hline

    Неразрывный пробел &
    $\sim$ \\ \hline

    Перенос строки &
    $\backslash \backslash$ \\ &
    ИЛИ \\ &
    $\backslash newline$ \\ \hline

    Полужирный шрифт &
    $\backslash textbf\{text\}$ \\ \hline

    Курсив &
    $\backslash textit\{test\}$ \\ \hline

    Подчёркивание &
    $\backslash underline\{text\}$ \\ \hline

    Эмфазис (выделение, по умолчанию переключает курсив) &
    $\backslash emph\{text\}$ \\ \hline

    Ввод формул (в той же строке) &
    $\$ formula \$$ \\ \hline

    Ввод формул (отдельным блоком) &
    $\backslash [ formula \backslash ]$ \\ \hline

    Сноска &
    $\backslash footnote\{some~text\}$ \\ \hline

    Огромный шрифт &
    $\backslash huge\{text\}$ \\ \hline

    Маленький шрифт &
    $\backslash tiny\{text\}$ \\ \hline

  \end{tabular}
\end{center}

\end{document}
