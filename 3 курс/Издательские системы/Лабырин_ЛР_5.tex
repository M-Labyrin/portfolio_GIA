\documentclass[12pt]{article}

\renewcommand{\familydefault}{\sfdefault}

\usepackage[T2A]{fontenc}
\usepackage[utf8]{inputenc}
\usepackage[russian]{babel}

\usepackage[left=2cm,right=2cm,
  top=2cm,bottom=2cm,bindingoffset=0cm]{geometry}

\setlength{\parindent}{0pt}
\setlength{\parskip}{1em}

\author{Лабырин Матвей, 1 подгруппа}
\title{Основы работы в \LaTeX}

\begin{document}

\maketitle

\tableofcontents

\section{Издательские системы}

\subsection{Издательская система \TeX}

\TeX~--- система компьютерной вёрстки, разработанная американским профессором информатики Дональдом Кнутом в целях создания компьютерной типографии.
В неё входят средства для секционирования документов, для работы с перекрёстными ссылками.
В частности, благодаря этим возможностям, \TeX~популярен в академических кругах, особенно среди математиков и физиков.

\subsection{Дональд Кнут}

Дональд Эрвин Кнут (англ. Donald Ervin Knuth, род. 10 января 1938 года, Милуоки, штат Висконсин) — американский учёный в области информатики.

Эмерит-профессор Стэнфордского университета, почетный доктор СПбГУ и других университетов, преподаватель и идеолог программирования, автор 19 монографий (в том числе ряда классических книг по программированию) и более 160 статей, разработчик нескольких известных программных технологий.
Автор всемирно известной серии книг, посвящённой основным алгоритмам и методам вычислительной математики, а также создатель настольных издательских систем \TeX~и METAFONT, предназначенных для набора и вёрстки книг научно-технической тематики (в первую очередь — физико-математических).

\subsection{Издательская система \LaTeX}

\LaTeX~--- наиболее популярный набор макрорасширений (или макропакет) системы компьютерной вёрстки \TeX, который облегчает набор сложных документов.
В типографском наборе системы \TeX~форматируется традиционно как \LaTeX.

Важно заметить, что ни один из макропакетов для \TeX’а не может расширить возможностей \TeX~(всё, что можно сделать в \LaTeX’е, можно сделать и в \TeX’е без расширений), но, благодаря различным упрощениям, использование макропакетов зачастую позволяет избежать весьма изощрённого программирования.

Пакет позволяет автоматизировать многие задачи набора текста и подготовки статей, включая набор текста на нескольких языках, нумерацию разделов и формул, перекрёстные ссылки, размещение иллюстраций и таблиц на странице, ведение библиографии и др.
Кроме базового набора существует множество пакетов расширения \LaTeX.
Первая версия была выпущена Лесли Лэмпортом в 1984 году; текущая версия, \LaTeX2$\varepsilon$, после создания в 1994 году испытывала некоторый период нестабильности, окончившийся к концу 1990-х годов, а в настоящее время стабилизировалась (хотя раз в год выходит новая версия).

\subsection{Лесли Лэмпорт}

Лесли Лэмпорт (англ. Leslie Lamport; 7 февраля 1941 года, Нью-Йорк) — американский учёный в области информатики, первый лауреат премии Дейкстры.
Разработчик \LaTeX~— популярного набора макрорасширений системы компьютерной вёрстки \TeX, исследователь теории распределённых систем, темпоральной логики и вопросов синхронизации процессов во взаимодействующих системах.
Лауреат Премии Тьюринга 2013 года.

Член Национальной академии наук США (2011), Национальной инженерной академии США (1991).

\section{Основные правила создания текстового документа}

Для создания документа в \LaTeX~нам достаточно указать тип документа в заголовке через команду $\char`\\documentclass\{*class*\}$, после чего мы можем указать все настройки документа, начиная от форматирования и задачи переменных окружения и заканчивая макросами, пользовательскими командами и целыми пакетами, делающими всю работу за нас ($\char`\\usepackage$).

Следом за подобным заголовком располагается сам документ преамбулой (то есть окружением $document$) при помощи $\char`\\begin\{document\}$ и $\char`\\end\{document\}$.

\end{document}
