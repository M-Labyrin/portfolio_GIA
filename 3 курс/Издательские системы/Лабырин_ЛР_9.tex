\documentclass[12pt]{article}

\usepackage[T2A]{fontenc}
\usepackage[utf8]{inputenc}
\usepackage[russian]{babel}

\renewcommand{\familydefault}{\sfdefault}

\usepackage[left=2cm,right=2cm,
  top=2cm,bottom=2cm,bindingoffset=0cm]{geometry}

\usepackage[skip=1em]{parskip}


% \usepackage{fontspec}
% \setmainfont{Arial}

% \usepackage{amsmath}
% \usepackage{amssymb}

\author{Лабырин М. С., 1 подгруппа}
\title{Создание таблиц в \LaTeX{}}

% \hfuzz=30pt

% chktex-file 44 (Чтобы линтер не ругался на стандартные таблички - он
% предлагает использовать пакет booktabs и НЕ использовать крайние линии)

\begin{document}

\maketitle

\section{Создание таблицы}

\subsection{Инструменты}

\begin{enumerate}
  \item Вкладка Помощник
  \item Быстрая таблица
\end{enumerate}

или

\begin{enumerate}
  \item \LaTeX{}
  \item Таблицы
\end{enumerate}

\subsection{Быстрая таблица}

\subsubsection{Шаг 1}

Выберите кол-во столбцов и кол-во строк\\
\begin{tabular}{|c|c|c|c|c|}
  \hline
  * & * & * & * & * \\
  \hline
  * & * & * & * & * \\
  \hline
  * & * & * & * & * \\
  \hline
  * & * & * & * & * \\
  \hline
  * & * & * & * & * \\
  \hline
\end{tabular}

\subsubsection{Шаг 3}

Объедините ячейки\\
\begin{tabular}{|c|c|c|c|c|}
  \hline
  № & Критерии & \multicolumn{3}{|c|}{Параметры}\\
  \hline
  * & * & выполнено полностью & выполнено частично & не выполнено \\
  \hline
  1 & * & * & * & * \\
  \hline
  2 & * & * & * & * \\
  \hline
  3 & * & * & * & * \\
  \hline
\end{tabular}

Объединить по вертикали

\begin{tabular}{|c|c|c|c|c|}
  \hline
  № & Критерии & \multicolumn{3}{|c|}{Параметры}\\
  \hline
  * & * & выполнено полностью & выполнено частично & не выполнено \\
  \hline
  1 & * & * & * & * \\
  \hline
  2 & * & * & * & * \\
  \hline
  3 & * & * & * & * \\
  \hline
\end{tabular}

\begin{tabular}{|c|c|}
  \hline
  * & * \\
  \hline
  * & * \\
  \hline
\end{tabular}

\end{document}
