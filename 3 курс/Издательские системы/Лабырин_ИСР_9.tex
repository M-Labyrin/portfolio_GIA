\documentclass[12pt]{article}

\usepackage[T2A]{fontenc}
\usepackage[utf8]{inputenc}
\usepackage[russian]{babel}

\renewcommand{\familydefault}{\sfdefault}

\usepackage[left=2cm,right=2cm,
  top=2cm,bottom=2cm,bindingoffset=0cm]{geometry}

\usepackage[skip=1em]{parskip}

\usepackage{multicol}

% \usepackage{fontspec}
% \setmainfont{Arial}

% \usepackage{amsmath}
% \usepackage{amssymb}

\author{Лабырин М. С., 1 подгруппа}
\title{Таблица Интегралов и дифференциалов}

% \hfuzz=30pt

% chktex-file 44 (Чтобы линтер не ругался на стандартные таблички - он
% предлагает использовать пакет booktabs и НЕ использовать крайние линии)

\renewcommand{\arraystretch}{2} % Хак для высоты строк

\begin{document}

\maketitle

\begin{multicols}{2}

  \section*{Интегралы}

  \begin{tabular}{ || l | l || }
    \hline
    Выражение & Эквивалент \\
    \hline

    $\displaystyle\int dx$ & $x + C$ \\[5pt]
    \hline

    $\displaystyle\int x^n \cdot dx$ & $\displaystyle \frac{x^{n + 1}}{n + 1} + C$, \\ &
    $n \neq -1$, $x > 0$ \\
    \hline

    $\displaystyle \int \frac{dx}{x}$ & $\ln |x| + C$ \\[5pt]
    \hline

    $\displaystyle \int a^x dx$ & $\displaystyle \frac{a^x}{\ln a} + C$ \\[5pt]
    \hline

    $\displaystyle \int e^x dx$ & $e^x + C$ \\[5pt]
    \hline
  \end{tabular}

  \columnbreak

  \section*{Дифференциалы}

  \begin{tabular}{ || l | l || }
    \hline
    Выражение & Эквивалент \\
    \hline

    $d(x^n)$ & $nx^{n - 1} dx$ \\
    \hline

    $d(a^x)$ & $a^x \cdot \ln a~dx$ \\
    \hline

    $d(e^x)$ & $e^x dx$ \\
    \hline

    $d(\log_a x)$ & $\displaystyle \frac{dx}{x \ln a}$ \\[5pt]
    \hline

    $d(\ln x)$ & $\displaystyle \frac{dx}{x}$ \\[5pt]
    \hline
  \end{tabular}

\end{multicols}

\end{document}
