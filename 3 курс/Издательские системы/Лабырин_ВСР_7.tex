\documentclass[12pt]{article}

\usepackage[T2A]{fontenc}
\usepackage[utf8]{inputenc}
\usepackage[russian]{babel}

\renewcommand{\familydefault}{\sfdefault}

\usepackage[left=2cm,right=2cm,
  top=2cm,bottom=2cm,bindingoffset=0cm]{geometry}

\usepackage[skip=1em]{parskip}

\author{Лабырин М. С., 1 подгруппа}
\title{Тема 7. Вариативное задание:\\Формулы сокращённого умножения}

\begin{document}

\maketitle

Квадрат суммы (\ref{ssum}):
\begin{equation}
  {(a + b)}^2 = a^2 + 2ab + b^2
  \label{ssum}
\end{equation}

Квадрат разности (\ref{sdif}):
\begin{equation}
  {(a - b)}^2 = a^2 - 2ab + b^2
  \label{sdif}
\end{equation}

Разность квадратов (\ref{dsqr}):
\begin{equation}
  a^2 - b^2 = (a - b) (a + b)
  \label{dsqr}
\end{equation}

Куб суммы (\ref{csum}):
\begin{equation}
  {(a + b)}^3 = a^3 + 3a^2 b + 3a b^2 + b^3
  \label{csum}
\end{equation}

Куб разности (\ref{cdif}):
\begin{equation}
  {(a - b)}^3 = a^3 - 3a^2 b + 3a b^2 - b^3
  \label{cdif}
\end{equation}

Сумма кубов (\ref{scub}):
\begin{equation}
  a^3 + b^3 = (a + b) (a^2 - ab + b^2)
  \label{scub}
\end{equation}

Разность кубов (\ref{dcub}):
\begin{equation}
  a^3 - b^3 = (a - b) (a^2 + ab + b^2)
  \label{dcub}
\end{equation}

\end{document}
