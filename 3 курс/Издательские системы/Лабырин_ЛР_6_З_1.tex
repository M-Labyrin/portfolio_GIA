% "Лабораторная работа 2"

\documentclass[a4paper,12pt]{article} % тип документа

% report, book

%  Русский язык
\usepackage[T2A]{fontenc}           % кодировка
\usepackage[utf8]{inputenc}         % кодировка исходного текста
\usepackage[english,russian]{babel}	% локализация и переносы

% Санс шрифт
\renewcommand{\familydefault}{\sfdefault}

% Нормальные поля
\usepackage[left=2cm,right=2cm,
  top=2cm,bottom=2cm,bindingoffset=0cm]{geometry}

% Абзацные отступы = некрасиво, поэтому так вот...
\usepackage[skip=1em]{parskip}

% Математика (не нужно ни разу, но раз тут было - я оставил)
\usepackage{amsmath,amsfonts,amssymb,amsthm,mathtools} 

% Глифы (тоже не нужно)
\usepackage{wasysym}

% Гиперссылки (исправил опечатки из шаблона)
\usepackage{hyperref}

% Заговолок
\author{Лабырин М. С., 1 подгруппа}
\title{Работа с текстом в \LaTeX}
\date{\today}

% Увеличить максимальную допустимую величину строки, чтобы Латех не ругался на вопросы в виде заголовков
\hfuzz=15pt


\begin{document} % начало документа

\maketitle

\section{Для чего предназначена издательская система \LaTeX?}

Издательская система \LaTeX{} предназначена для набора и вёрстки текстов любой сложности в самом совместимом и доступном формате --- текстовом (plain text).

\section{В каких случаях рационально её использовать?}

Как правило, систему \LaTeX{} используют, когда требуется записывать много формул и спецсивмолов и имеется необходимость в возможностях вёрстки большей, нежели предоставляют нам текстовые процессоры вроде MS Word или LibreOffice Writer.

От себя могу отметить, что \LaTeX{} также имеет смысл использовать вместо любого текстового процессора тем, кто привык работать в таких редакторах, как Vim и Emacs.
Любая задача в таком случае решается в б\'oльшем комфорте, с б\'oльшей скоростью и б\'oльшими возможностями.

\section{Какие преимущества имеет работа в этой системе?}

Прежде всего --- это то, что пишется \TeX{} документ простым текстом.
У простого текста есть ряд очевидных преимуществ:
\begin{enumerate}
  \item Простой текст максимально совместим --- его откроет любой редактор.
  \item Простой текст можно обрабатывать через sed или другие модульные внешние утилиты.
  \item Простой текст максимально прост и понятен как таковой, сам по себе он не обладает странными и убогими системами форматирования.
  \item Простой текст независим ни от какой компании и их программых пакетов --- он всегда будет оставаться свободным и открытым.
  \item Простой текст без проблем можно переконвертировать в любой другой формат без потерь.
    Для других форматов это зачастую неактуально.
\end{enumerate}

Также безусловным преимуществом \LaTeX{} является его модульность и, как следствие, функциональность --- чего бы Вы не пожелали и в чём бы ни нуждались --- кто-то уже почти наверняка написал пакет, делающий это за Вас.

В Латеке многие вещи автоматизированы <<из коробки>> либо в пакетах.
Например, нумерация чего угодно происходит автоматически, а ссылки всегда указывают на верно указанный номер.\\
Можете вспомнить с болью на душе, как иногда выбешивает, когда тот или иной текстовый процессор ошибается с нумерацией, а чтобы это починить, приходится возиться с этим столько, сколько имеется таких нумерованных элементов.


\section{Какие сложности могут возникнуть при работе в этой системе?}

\LaTeX{} имеет довольно высокий порог освоения (как и любая другая стоящая себя система), однако однажды изучив его, Вы буквально станете сверхчеловеком в плане форматирования и вёрстки публикаций.

У пользователей, которые используют плебейские редакторы, могут возникнуть проблемы с удобством набора на Латеке.
Для прекрасных пользователей редакторов кода вроде Vim и Emacs таких проблем не возникает.
Да и раз уж Вы самовольно решили изучить Латек --- то Вы наверняка уже знакомы с данными редакторами в той или иной мере.

Вывод --- хватит торчать на своих вордах и ай-да ко мне на вим!

\section{Какие недостатки отмечают пользователи при работе с этой системой?}

Им сложно.
Они не хотят потратить немного своего невероятно важного времени на изучение мощнейшего инструмента набора и форматирования текста, который сэкономил бы им куда больше времени и нервов.
Они предпочитают пользоваться тем, что дают, ни в какую не пытаясь самосовершенствовать как себя, так и своё рабочее окружение.

Я, как пользователь, не могу отметить ни одного действительно важного недостатка --- все они так или иначе решаются тем или иным способом.
В том, что они есть (например, некоторые проблемы с символами Юникода), виноват исключительно я --- тем, что не умею их решать.

В этом и проявляется изумительное свойство подобных систем: зачастую единственное, что Вас ограничивает в работе с ними --- это Вы сами.
Системы с потенциалом, б\'oльшим, нежели потенциал самого пользователя, позволяют последним развиваться.

\section{Возможность использовать упомянутые в задании функции}

\begin{center}
  \href{https://www.google.com/}{Гугл, чтобы искать справочную информацию по \LaTeX}
\end{center}

\begin{flushright}
  \huge{Справа?}\footnote{Справа.}
\end{flushright}

\begin{itemize}
  \item А где цифры?
  \item А нигде --- это маркированный список, а не нумерованный.
  \item \tiny{Ладно..}
\end{itemize}

\end{document} % конец документа
