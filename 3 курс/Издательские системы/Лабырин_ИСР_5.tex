\documentclass[12pt]{article}

\usepackage[T2A]{fontenc}
\usepackage[utf8]{inputenc}
\usepackage[russian]{babel}

\renewcommand{\familydefault}{\sfdefault}

\usepackage[left=2cm,right=2cm,
  top=2cm,bottom=2cm,bindingoffset=0cm]{geometry}

\usepackage[skip=1em]{parskip}

\author{Лабырин Матвей, 1 подгруппа}
\title{Команды создания текстового документа в \LaTeX}

% chktex-file 18

\begin{document}

\maketitle

\tableofcontents

\section{Команды создания документа}

При создании документа в \LaTeX, в условном "заголовке" документа мы указываем следующие команды, задающие параметры генерации документа:

$\backslash documentclass[params]\{class\}$ --- задаёт класс документа \emph{class} с заданными в \emph{params} параметрами.
Например, $\backslash documentclass[12pt]\{article\}$ задаст класс \emph{article} с параметром величины шрифта в \emph{12pt}.

$\backslash usepackage[params]\{package\}$ --- подключает к документу пакет \emph{package} с заданными в \emph{params} параметрами.
Например, $\backslash usepackage[russian]\{babel\}$ подключит пакет работы с языками \emph{babel} с параметром-флагом \emph{russian}, т.е. для русского языка.

$\backslash author\{author\}$ --- задаёт автора документа (по умолчанию отсутствует).

$\backslash title\{title\}$ --- задаёт название документа (по умолчанию отсутствует).

$\backslash date\{date\}$ --- задаёт дату документа (по умолчанию равна текущей дате, т.е. команде $\backslash today$).

$\backslash begin\{document\} \cdots \backslash end\{document\}$ --- команды задачи окружения, которые задают тело документа (т.е. само его содержание). Всё, что внутри этого окружения, будет так или иначе отображаться в самом документе.

\section{Команды создания текста в документе}

Внутри окружения \emph{document} располагается само тело документа.
По умолчанию оно совершенно пустое, что совершенно логично --- ведь мы там ещё ничего не написали.
Следующие команды позволят автоматически сгенерировать внутри документа некоторые тривиальные вещи, основываясь на классе документа и на его параметрах:

$\backslash maketitle$ --- генерирует заголовок документа, отображая в нём по умолчанию название, автора и дату документа (для класса \emph{article}).

$\backslash tableofcontents$ --- генерирует содержание документа из указанных в нём глав.

$\backslash section\{title\}$ (а также $\backslash subsection\{title\}$, $\backslash subsubsection\{title\}$) --- генерирует не только заголовок главы/подглавы/подподглавы, но и сам структурный элемент текста, позволяя, например, команде $\backslash tableofcontents$ генерировать содержание.
Номера, что очень удобно, генерируются автоматически.

\section{Заключение}

Приведённых выше команд уже достаточно для создания полноценного текстового документа без форматирования.

Спасибо за внимание!

\end{document}
