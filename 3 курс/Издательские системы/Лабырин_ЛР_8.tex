\documentclass[12pt]{article}

\usepackage[T2A]{fontenc}
\usepackage[utf8]{inputenc}
\usepackage[russian]{babel}

\renewcommand{\familydefault}{\sfdefault}

\usepackage[left=2cm,right=2cm,
  top=2cm,bottom=2cm,bindingoffset=0cm]{geometry}

\usepackage[skip=1em]{parskip}

\usepackage{amsmath}
\usepackage{amssymb}

\author{Лабырин М. С., 1 подгруппа}
\title{Диакритические знаки, матрицы\\Лабораторная работа 4}

% \hfuzz=30pt

\begin{document}

\maketitle

\section{Диакритические знаки}

\subsection{Надстрочные}

\[\dot{x} = 0\]
\[\tilde{a} = \bar{b}\]
\[\tilde{a} = \overline{bcde}\]

широкая тильда

\[\widetilde{afgh} = \overline{bcde}\]

многоточие $\cdots$

\subsection{Векторы}
Вектор а имеет координаты (0;3;4)
\[\overrightarrow{a}(0;3;4)\]

Запись вектора жирным шрифтом, а не стрелкой сверху
\[\overrightarrow{a} = \mathbf{a}\]

\subsection{Фигурная скобка}
\[\underbrace{1 + 2 + \cdots + n} = N\]
\[\underbrace{1 + 2 + \cdots + n}_n = N\]

\begin{equation}
  \underbrace{1 + 2 + \cdots + n} = N
\end{equation}

\begin{equation}
  \underbrace{1 + 2 + \cdots + n}_n = N
\end{equation}

\begin{equation}
  \overbrace{1 + 2 + \cdots + n}^n = N
\end{equation}

\subsection{Написание условия перехода над знаком}
команда \textbf{stackrel}\\
Например:
\[ (x - 1) (x + 1) > 0 \stackrel{x > 0}{\longleftrightarrow} (x - 1) > 0 \]

\section{Буквы других алфавитов}

\[ \sin \alpha = 0 \]
\[ \omega = \frac{2 \pi}{T} \]
непривычный вид\\
$\epsilon$\\
$\phi$

как в учебниках\\
$\varepsilon$\\
$\varphi$

\subsection{Математические шрифты}
много\\
один из них \textbf{mathbb}
находятся во вкладке Математика/Математические шрифты
\[ x \in R \]
\[ x \in \mathbb{R} \]

\subsection{Кириллические символы}
используется команда \textbf{text}
\[ m_{\text{груза}} = 15~\text{кг} \]

Для пробела между обозначением величины и её численным значением необходимо
использовать тильду \char`\~

\section{Выравнивание формул}
окружение \textbf{aligned}\\
определяет выравнивание амперсант \&

\section{Группировка формул}
\begin{equation}
  \begin{aligned}
    4 & \times a = 8 \\
    -5 & \times b = 10 \\
    -10 & \times c = 110 \\
  \end{aligned}
\end{equation}

\subsection{Системы уравнений}

\[
  \left \{
    \begin{aligned}
      4 & \times a = 8 \\
      -5 & \times b = 10 \\
      -10 & \times c = 110 \\
    \end{aligned}
  \right.
\]

\[
  \left.
    \begin{aligned}
      4 & \times a = 8 \\
      -5 & \times b = 10 \\
      -10 & \times c = 110 \\
    \end{aligned}
  \right \}
\]

\[
  \left.
    \begin{aligned}
      4 & \times a = 8 \\
      -5 & \times b = 10 \\
      -10 & \times c = 110 \\
    \end{aligned}
  \right \} \Rightarrow -12ab = 24
\]

\section{Матрицы}
Создаются за счёт окружения \textbf{matrix}

\subsection{Матрица в круглых скобках}

\[
  \begin{pmatrix}
    a_{11} & a_{12} & a_{13} \\
    a_{21} & a_{22} & a_{23} \\
    a_{31} & a_{32} & a_{33}
  \end{pmatrix}
\]

\subsection{Матрица в квадратных скобках}

\[
  \begin{bmatrix}
    a_{11} & a_{12} & a_{13} \\
    a_{21} & a_{22} & a_{23} \\
    a_{31} & a_{32} & a_{33}
  \end{bmatrix}
\]

\[
  \begin{Vmatrix}
    a_{11} & a_{12} & a_{13} \\
    a_{21} & a_{22} & a_{23} \\
    a_{31} & a_{32} & a_{33}
  \end{Vmatrix}
\]

\subsection{Определитель}

\[
  \begin{vmatrix}
    a_{11} & a_{12} & a_{13} \\
    a_{21} & a_{22} & a_{23} \\
    a_{31} & a_{32} & a_{33}
  \end{vmatrix}
\]

\end{document}
