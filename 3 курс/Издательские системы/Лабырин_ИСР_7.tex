\documentclass[12pt]{article}

\usepackage[T2A]{fontenc}
\usepackage[utf8]{inputenc}
\usepackage[russian]{babel}

\renewcommand{\familydefault}{\sfdefault}

\usepackage[left=2cm,right=2cm,
  top=2cm,bottom=2cm,bindingoffset=0cm]{geometry}

\usepackage[skip=1em]{parskip}

\author{Лабырин М. С., 1 подгруппа}
\title{Команды набора формул в \LaTeX}

\hfuzz=30pt

\begin{document}

\maketitle

\begin{center}
  % chktex-file 44
  \begin{tabular}{ || l | l || }

    \hline
    Назначение команды & Вид (написание) команды \\ \hline
    \hline

    Включенная формула &
    $\$ \cdots \$$ \\ \hline

    Выключенная формула &
    $\$\$ \cdots \$\$$ \\ &
    $\backslash[~\cdots \backslash]$ \\ &
    $\backslash begin\{equation\} \cdots \backslash end \{equation\}$ \\ \hline

    Дробь &
    $\backslash frac\{a\}\{b\}$ \\ \hline

    Авторазмер скобок &
    $\backslash left (~\ldots \backslash right )$ \\ &
    $\backslash left [~\ldots \backslash right ]$ \\ &
    $\backslash left \{~\ldots \backslash right \}$ \\ \hline

    Индекс &
    $x\_n$ \\ &
    $x\_\{n - 1\}$ \\ \hline

    Показатель (степень) &
    $x\string^i$ \\ &
    $x\string^\{i - 1\}$ \\ \hline

    Умножение ($\cdot$) &
    $\backslash cdot$ \\ \hline

    Точки ($\cdots$) &
    $\backslash cdots$ \\ \hline

    Знак неравенства ($\neq$) &
    $\backslash neq$ \\ \hline

    Знак призлизительного равенства ($\approx$) &
    $\backslash approx$ \\ \hline

    Сумма ($\sum$) &
    $\backslash sum$ \\ \hline

    Интеграл ($\int$) &
    $\backslash int$ \\ &
    $\backslash int \backslash limits\_\{a\}\{b\}$ \\ \hline

    Бесконечность ($\infty$) &
    $\backslash infty$ \\ \hline

    Стандартные функции &
    $\backslash sin$ \\ &
    $\backslash ln$ \\ &
    и т.п. \\ \hline

  \end{tabular}
\end{center}

\end{document}
